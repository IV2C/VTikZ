\documentclass[tikz,border=5]{standalone}
\usepackage{tikz}
\usetikzlibrary{positioning}

\begin{document}
    \begin{tikzpicture}[%
        helpGrid/.style = {thin, gray!40},
        ] % End tikzpicture options

        %- X - Axis (Months) at the top
        % Months 2020
        \node[align=center] (Jan20) at (0,11) {Jan \\ 2020}; 
        \node[] (Apr20)  at (2,11) {Apr};
        \node[] (Jul20)   at (4,11) {Jul};
        \node[] (Oct20) at (6,11) {Oct};
        % Months 2021
        \node[align=center] (Jan21) at (8,11) {Jan \\ 2021}; 
        
        %- Corresponding other nodes
        % Column 1 (Jan 2020)
        \node[] (a1) at (0,10) {A};
        \node[] (b1) at (0,9) {B};
        \node[] (c1) at (0,8) {C};
        
        % Column 2 (Apr 2020)
        \node[] (d2) at (2,10)  {D};
        \node[] (a2) at (2,9)  {A};
        % Arrows connecting column 1 and 2
        \draw[-latex] (a1) to [out = 0, in = 180] (a2);
        
        % Column 3 (Jul 2020)
        \node[] (e3) at (4,7) {E};
        \node[] (b3) at (4,6) {B};
        \node[] (c3) at (4,5) {C};
        \node[] (f3) at (4,4) {F};
        
        % Arrows connecting column 2 and 3
        \draw[-latex] (b1) to [out = 0, in = 180] (b3);
        \draw[-latex] (c1) to [out = 0, in = 180] (c3);
        
        %- Column 4 (Oct)
        % Nothing
        
        % Column 5 (Jan 2021)
        \node[] (g5) at (8,10) {G};
        \node[] (a5) at (8,8) {A};
        \node[] (e5) at (8,7) {E};
        \node[] (b5) at (8,6) {B};
        \node[] (c5) at (8,3) {C};
        \node[] (f5) at (8,2) {F};
        
        % Arrows connecting columns 3 and 5
        \draw[-latex] (a2) to [out = 0, in = 180] (a5);
        \draw[-latex] (e3) -- (e5);
        \draw[-latex] (b3) -- (b5);
        \draw[-latex] (c3) to [out = 0, in = 180] (c5);
        \draw[-latex] (f3) to [out = 0, in = 180] (f5);
        
    \end{tikzpicture}
\end{document}